
\newcommand{\rb}[1]{\raisebox{1.5ex}{#1}}
 \newcommand{\trace}{\mathrm{trace}}



%\def\real{\mathbb{R}}
\newcommand{\real}{\mathbb R}
\def\reals{\mathbb{R}} %????
\newcommand{\nat}{\mathbb N}   
\newcommand{\whole}{\mathbb Z}    
\newcommand{\cp}{\mathbb C}    
\newcommand{\rat}{\mathbb Q} 
\newcommand{\im}{\mathbb i}
\newcommand{\impart}[1]{\mathrm{imag} \left( #1 \right)}
\newcommand{\repart}[1]{\mathrm{real} \left( #1 \right)}
%\newcommand{\angle}[1]{\mathrm{angle} (#1 )}

% Hack for a bold integral sign
\newcommand{\boldint}{%
    \ThisStyle{%
        \setbox0=\hbox{$\SavedStyle\int$}%
        \ooalign{%
            \hfil\scalebox{1.05}[1]{$\SavedStyle\int$}\hfil\cr
            \hfil\scalebox{1.1}[1]{$\SavedStyle\int$}\hfil\cr
            \hfil\scalebox{1.15}[1]{$\SavedStyle\int$}\hfil\cr
        }%
    }%
}
\newcommand{\customdot}[1]{\stackon[1.5pt]{$#1$}{\tiny\textbullet}}



\newcommand{\ds}{\displaystyle}
\newcommand{\mf}[2]{\frac{\ds #1}{\ds #2}}
\newcommand{\book}[2]{{Boyd, Page~#1, }{Prob.~#2}}
\newcommand{\Kuttler}[1]{Kuttler 2017, A First Course in Linear Algebra, Page(s)~#1}
\newcommand{\spanof}[1]{\mathrm{span} \{ #1 \}}
\newcommand{\colspanof}[1]{\mathrm{col~span} \{ #1 \}}
\newcommand{\nullspace}[0]{\mathrm{null}}
\newcommand{\nullity}[0]{\mathrm{nullity}}
\newcommand{\range}[0]{\mathrm{range}}
\newcommand{\cost}[0]{\mathrm{f}}
\newcommand{\rank}[0]{\mathrm{rank}}
\newcommand{\diag}[0]{\mathrm{diag}}
\newcommand{\prem}[0]{\mathrm{p_{\rm rem}}}
\newcommand{\bigO}[0]{\mathrm{O}}
\newcommand{\ustep}[0]{u_{\rm stp}}

\newcommand{\sign}[0]{\mathrm{sign}}
\newcommand{\acosh}[0]{\mathrm{acosh}}
\newcommand{\asinh}[0]{\mathrm{asinh}}
\newcommand{\atan}[0]{\mathrm{atan}}
\lstset{
  literate={divSign}{{$\div$}}1
}


\newcommand{\protextjwg}[1]{\protect\( #1 \protect\)}


\newcommand{\cprod}[0]{\cdot \ldots \cdot}



\newcommand{\chatgpt}[0]{\textbf{ChatGPT}}


%%%%%%%%%%%%%%%%%%%%From the ECE Contrtol Book
\newcommand{\bit}[1]{\textcolor{blue}{\it  #1 }}
\newcommand{\emphas}[1]{\textcolor{blue}{  #1 }}



\newcommand{\reviewbox}[1]{ \begin{tcolorbox}
\begin{minipage}{0.9\textwidth}
#1 
\end{minipage}
\end{tcolorbox}
}


\newcommand{\UnitStep}[0]{u{\protectjwg{_{\rm s}}}}

%%%%%%%%%%%%%%%%%%%%

%%%%%%%%%%%%%%%%%%%
% Bruce added below
%%%%%%%%%%%%%%%%%%%
\newcommand{\transpose}{\mathsf{T}}
\DeclareDocumentCommand{\zeros}{ O{} }{\textbf{0}_{#1}}
\newcommand\inv[1]{#1\raisebox{1.15ex}{$\scriptscriptstyle-\!1$}}
\newcommand{\Exp}{\mathrm{Exp}}
\newcommand{\Log}{\mathrm{Log}}
\newcommand{\arcos}{\mathrm{arccos}} 
\newcommand{\sinc}{\mathrm{sinc}}  
\newcommand{\rect}{\mathrm{rect}}
\newcommand{\tri}{\mathrm{tri}}
\renewcommand{\im}{\mathop{\bm{\mathrm i}}}

% \newcommand{\company}{ \textcolor{blue}{\bf Algorithms in Motion}\textregistered}
% }

\newcommand{\company}{\textcolor{blue}{\bfseries Algorithms in Motion}\textsuperscript{\textregistered}}


%%\href{https://www.dropbox.com/s/1pzva81t4shnd22/ROB_101_Textbook_11April2023.pdf?dl=0}{textbook}

\newtheorem{remark}{Remark}
%%%%%%%%%%%%%%%%%%%
% Bruce added above
%%%%%%%%%%%%%%%%%%%



%\newcommand{\dim}[0]{\mathrm{dim}}
 \newcommand{\cov}{\mathrm{cov}}
 \newcommand{\E}{\mathcal{E}}
\parindent 0pt

\newcommand{\notimplies}{%
  \mathrel{{\ooalign{\hidewidth$\not\phantom{=}$\hidewidth\cr$\implies$}}}}


\DeclareMathOperator*{\argmin}{arg\,min}
\DeclareMathOperator*{\argmax}{arg\,max}

\newcommand{\rbf}[1]{\text{\rmfamily\bfseries #1}}

\usepackage{upgreek}
\newcommand{\rbfalpha}{\bm{\upalpha}}

\newcommand{\hilight}[1]{\colorbox{yellow}{#1}} % to use: \hl{this is some highlighted text}
\newcommand{\red}[1]{{\textcolor{red}{#1}}}
\newcommand{\textbfred}[1]{{\textcolor{red}{\bf #1}}}


\newcommand{\blue}[1]{{\textcolor{blue}{#1}}}
\newcommand{\jwg}[1]{[{\textbf{\textcolor{red}{JWG: #1}}}]}
\newcommand{\kj}[1]{[{\textbf{\textcolor{teal}{KJ: #1}}}]}
\newcommand{\bh}[1]{[{\textbf{\textcolor{blue}{Bruce: #1}}}]}

\newcommand{\bprp}{{\bf BlackPen}\textcolor{red}{\bf RedPen}}
\newcommand{\threebb}{{\textcolor{blue}{\bf 3Blue}\textcolor{brown}{\bf 1Brown}}}


\newcommand\RED{\color{red}}
\newcommand\BLUE{\color{blue}}


\definecolor{darkblue}{rgb}{0,0,0.5}
\definecolor{lightblue}{rgb}{0.88,1,1}
%\definecolor{brightblue}{rgb}{0.0, 0.6, 1.0} % Adjust the RGB values to make the color brighter if necessary
\definecolor{jwgbluegray}{rgb}{0.8, 0.95, 0.95} % Light grayish blue

%% for the empheq environment from Fawwaz Ulabi
\definecolor{none}{cmyk}{0.0, 0.0, 0.0, 0.0}
% \newcommand*\bluebox[1]{%
%   \colorbox{jwgbluegray}{\hspace{1em}#1\hspace{1em}}}
 %% for the empheq environment
% Redefine the \bluebox command to use TikZ for drawing a frame around the content
\definecolor{brightblue}{RGB}{0, 0, 239} % More vibrant bright blue for the frame

% Redefine the \bluebox command to use TikZ for drawing a more prominent frame around the content
\newcommand*\bluebox[1]{%
  \tikz[baseline=(X.base)]\node [draw=brightblue, fill=jwgbluegray, very thick, rectangle, inner sep=2mm, rounded corners=2pt] (X) {#1};%
}

%   % Jessy's modified commands
% \newcommand{\emstat}[1]{\begin{center} \fcolorbox{none}{ltpurple}{%   %%JWG
%  \begin{minipage}{525pt}#1\end{minipage}}\bigskip \end{center}}
%  %% \begin{minipage}{450pt}#1\end{minipage}}\bigskip \end{center}} %% original value

% Edits on 2 Feb 2024
\newtcolorbox{emstatbox}[1][]{
  breakable,
  %colback=ltpurple,
  colback=jwgbluegray,
  colframe=brightblue, %%none,
  boxsep=0pt,
  left=2pt,
  right=2pt,
  top=5pt,
  bottom=5pt,
  arc=0pt,
  outer arc=0pt,
  boxrule=2pt, %0pt,
  #1
}

\newcommand{\emstat}[1]{\begin{center}\begin{emstatbox}#1\end{emstatbox}\end{center}}

 
\definecolor{note}{rgb}{0.3,0.7,0.25}
\definecolor{rephase}{rgb}{0.15,0.7,0.15}
\definecolor{bag}{rgb}{0.6,0.6,0.2}

\newcommand\x{\times}
\newcommand\bigzero{\makebox(0,0){\text{\huge0}}}
\newcommand*{\bord}{\multicolumn{1}{c|}{}}
\newcommand*{\bordl}{\multicolumn{1}{|c}{}}

\newcommand{\solution}{\textbf{Solution:~}}
\newcommand{\solutions}{\textbf{Solutions:~}}

\newcommand*{\Ans}{\textcolor{blue}{\bf Ans.~}}
\newcommand*{\Qed}{\hfill $\blacksquare$}

\newcommand\doverline[1]{\ThisStyle{%
  \setbox0=\hbox{$\SavedStyle\overline{#1}$}%
  \ht0=\dimexpr\ht0-.15ex\relax% CHANGE .15 TO AFFECT SPACING
  \overline{\copy0}%
}}

\newcommand{\rob}{\textbb{ROB 101}}

% \newcommand*\circled[1]{\tikz[baseline=(char.base)]{
%     \node[shape=circle, draw, inner sep=1pt, 
%         minimum height={\f@size*1.6},] (char) {\vphantom{WAH1g}#1};}}
\makeatother




%\maketitle


% \newtheorem{thm}{Theorem}
% \numberwithin{thm}{chapter}
% \newtheorem{prop}{Proposition}
% \numberwithin{prop}{chapter}
% \newtheorem{lemma}{Lemma}
% \numberwithin{lemma}{chapter}
% \newtheorem{claim}{Claim}
% \numberwithin{claim}{chapter}
% \newtheorem{rem}{Remark}
% \numberwithin{rem}{chapter}
% \newtheorem{def}{Definition}
% \numberwithin{def}{chapter}

\newtheorem{thm}{Theorem}
\numberwithin{thm}{chapter}
\newtheorem{example}[thm]{Example}
\newtheorem{nonexample}[thm]{Non-Example}
\newtheorem{prop}[thm]{Proposition}
\newtheorem{lem}[thm]{Lemma}
\newtheorem{cor}[thm]{Corollary}
\newtheorem{claim}[thm]{Claim}
\newtheorem{rem}[thm]{Remark}
\newtheorem{definition}[thm]{Definition}
\newtheorem{notation}[thm]{Notation}
\newtheorem{fact}[thm]{Fact}
\newtheorem{notvocab}[thm]{Notation and Vocabulary}
\newtheorem{question}[thm]{Question}
\newtheorem{recall}[thm]{Recall}
\newtheorem{exercise}[thm]{Exercise}
\newtheorem{summary}[thm]{Summary}

% This makes the table counter the same as the thm counter
\makeatletter
\let\c@table\c@thm
\makeatother
%
\newcounter{keyfacts}
\newtheorem{keyfact}[thm]{Key Fact}

\definecolor{lightgold}{rgb}{0.933, 0.867, 0.510} % Define a color for the light gold
\definecolor{logictitlecolor}{rgb}{0.000, 0.447, 0.698} % Define a color for the title bar
\definecolor{navy}{RGB}{0,0,128}


% Define new tcolorbox environment 'thmColor'
\newtcbtheorem[use counter=thm, number within=chapter]{thmColor}{Theorem}%
{sharp corners, colback=green!30, colframe=green!80!blue, breakable, fonttitle=\bfseries}{thm}

\newtcbtheorem[use counter=thm, number within=chapter]{propColor}{Proposition}%
{sharp corners, colback=green!30, colframe=green!80!blue, breakable, fonttitle=\bfseries}{thm}

\newtcbtheorem[use counter=thm, number within=chapter]{corColor}{Corollary}%
{sharp corners, colback=green!30, colframe=black, breakable, fonttitle=\bfseries}{thm}

\newtcbtheorem[use counter=thm, number within=chapter]{methodColor}{Method}%
{sharp corners, colback=gray!30, colframe=blue!80!black, breakable, fonttitle=\bfseries}{thm}

\newtcbtheorem[use counter=thm, number within=chapter]{observationColor}{Observation}%
{sharp corners, colback=orange!50, colframe=orange!70!black, breakable, fonttitle=\bfseries}{thm}

\newtcbtheorem[use counter=thm, number within=chapter]{factColor}{Fact}%
{sharp corners, colback=orange!50, colframe=orange!70!black, breakable, fonttitle=\bfseries}{thm}

\newtcbtheorem[use counter=thm, number within=chapter]{funColor}{Secrets of the Arcane}%
{sharp corners, colback=magenta!50, colframe=magenta!100!black, coltitle=black, breakable, fonttitle=\bfseries}{thm}

\newtcbtheorem[use counter=thm, number within=chapter]{logicColor}{Logic Principle }%
{sharp corners, colback=lightgold, colframe=black, coltitle=white, breakable, fonttitle=\bfseries}{thm}

% \begin{tcolorbox}[colback=mylightblue, title = {\bf Defining Functions in a Careful Manner}, breakable]

% \begin{definition} What is a function?     

% blah blah 

% \end{definition}
% \end{tcolorbox}

%% For Control Chapter

%\newtheorem{example}[thm]{Example}
%\numberwithin{example}{chapter}
\newtheorem{revquest}[thm]{\textcolor{red}{\bf Review Question:}}
%\numberwithin{question}{chapter}



