
\href{https://courses.lumenlearning.com/precalculus/chapter/inverse-trigonometric-functions/}{Lumen Learning Trig Functions}

\begin{table}
\centering
\begin{tabular}{|c|c|c|c|c|c|}
\hline
\textbf{Name} & \textbf{Usual notation} & \textbf{Definition} & \textbf{Domain of \(x\)} & \textbf{Range of usual principal } & \textbf{Range of usual principal} \\
 & &  & \textbf{for real result} & \textbf{value (radians)} & \textbf{ value (degrees)} \\
\hline
arcsine & \(y=\arcsin(x)\) & \(x = \sin(y)\) & \(-1\leq x\leq 1\) & \(-\frac{\pi}{2}\leq y\leq \frac{\pi}{2}\) & \(-90^{\circ}\leq y\leq 90^{\circ}\) \\
\hline
arccosine & \(y=\arccos(x)\) & \(x = \cos(y)\) & \(-1\leq x\leq 1\) & \(0\leq y\leq \pi\) & \(0^{\circ}\leq y\leq 180^{\circ}\) \\
\hline
arctangent & \(y=\arctan(x)\) & \(x = \tan(y)\) & all real numbers & \(-\frac{\pi}{2}< y <\frac{\pi}{2}\) & \(-90^{\circ}< y <90^{\circ}\) \\
\hline
\end{tabular}
\caption{Domain and Range of the Most Common Inverse Trigonometric Functions}
\label{tab:CommonTrigFunctions}
\end{table}


\begin{table}[h]
\centering
\begin{tabular}{|l|l|l|l|l|l|l|l|l|l|l|}
\hline
\textbf{Name} & \textbf{Symbol} & & \textbf{Domain} & & \textbf{Image/Range} & \textbf{Inverse function} & & \textbf{Domain} & & \textbf{Image of principal values} \\
\hline
sine & $\sin$ & : & $\mathbb{R}$ & $\to$ & $[-1, 1]$ & $\arcsin$ & : & $[-1, 1]$ & & \\
\hline
\end{tabular}
\caption{Your Caption}
\label{tab:MoreTrig}
\end{table}




\begin{table}[h]
\centering
\caption{Function Properties}
\begin{tabular}{|l|l|l|l|l|l|}
\hline
\multicolumn{2}{|c|}{\textbf{Name}} & \multicolumn{2}{c|}{\textbf{Domain}} & \multicolumn{2}{c|}{\textbf{Range}} \\ \hline
Function & Symbol & \multicolumn{1}{c|}{\textbf{Symbol}} & \multicolumn{1}{c|}{\textbf{Domain}} & \multicolumn{1}{c|}{\textbf{Symbol}} & \multicolumn{1}{c|}{\textbf{Range}} \\ \hline
Sine & $\sin$ & : & $\mathbb{R}$ & $\to$ & $[-1, 1]$ \\ \hline
Inverse Sine & $\arcsin$ & : & $[-1, 1]$ & $\to$ & $\left[-\frac{\pi}{2}, \frac{\pi}{2}\right]$ \\ \hline
Cosine & $\cos$ & : & $\mathbb{R}$ & $\to$ & $[-1, 1]$ \\ \hline
\end{tabular}
\end{table}



% \begin{lstlisting}[language=Julia,style=mystyle]

% \end{lstlisting}
% \textbf{Output} 
% \begin{verbatim}

% \end{verbatim}

% \begin{lstlisting}[language=Julia,style=mystyle]

% \end{lstlisting}
% \textbf{Output} 
% \begin{verbatim}

% \end{verbatim}

% \begin{lstlisting}[language=Julia,style=mystyle]

% \end{lstlisting}
% \textbf{Output} 
% \begin{verbatim}

% \end{verbatim}

% \begin{lstlisting}[language=Julia,style=mystyle]

% \end{lstlisting}
% \textbf{Output} 
% \begin{verbatim}

% \end{verbatim}

% \begin{lstlisting}[language=Julia,style=mystyle]

% \end{lstlisting}
% \textbf{Output} 
% \begin{verbatim}

% \end{verbatim}

% \begin{lstlisting}[language=Julia,style=mystyle]

% \end{lstlisting}
% \textbf{Output} 
% \begin{verbatim}

% \end{verbatim}

