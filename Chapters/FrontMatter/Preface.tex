
In the ever-evolving landscape of engineering education, calculus stands as a foundational pillar, yet its teaching methods have remained largely unchanged for decades. This textbook, a trailblazing effort in calculus education, is born out of a deep-seated need for reform. Since my tenure at the University of Michigan began in 1987, I have witnessed firsthand the challenges and limitations of the traditional calculus sequence in mathematics education, particularly within our College of Engineering (CoE).

Traditional calculus courses struggle to incorporate relevant, real-world examples. My colleagues in mathematics have been given an impossible job: present realistic examples in domains far from their fields of expertise. Students see right through this. The disconnect between mathematical theory and its real-world uses hinders the learning process and fails to fully engage students in the practical applications of calculus. A second core issue lies not in the examples used but in the selection of material and the overemphasis on lengthy, manual calculations that lack meaningful context. 

This book represents a radical departure from conventional approaches, inspired by my experience developing and teaching ROB 101 \textit{Computational Linear Algebra}. It aims to breathe life into mathematics through programming and realistic examples. In collaboration with my colleague Prof. Chad Jenkins, we have reimagined the calculus curriculum, placing integration at the beginning of the learning process. This approach, starting with sums and progressing to integration, is not only more intuitive but also lends itself well to programming applications, such as calculating the change in position of mobile robots from velocity curves.

Following integration, the book delves into differentiation (both single-variable and multivariable) and its myriad applications, culminating in a comprehensive chapter that instructors can tailor to their needs. Traditional methods of computing integrals, such as antiderivatives and substitution methods, are then introduced, and framed within the context of both manual calculations and modern symbolic tools.

The journey continues through improper integrals, ordinary differential equations (ODEs), Laplace Transforms, and Feedback Control, encompassing a breadth of topics rarely seen in a single calculus course. This ambitious curriculum is not just theoretical; it is being put to the test in a pilot program, reminiscent of the successful pilot of the year-one first-semester course, ROB 101, which included LU Factorization in its fifth week.

\subsubsection{Integration with Michigan Robotics' Curriculum}

This course is carefully designed to align with the needs of higher-level courses in Robotics. For instance, ROB 310 \textit{Robot Sensors and Signals} uses Laplace transforms, ROB 311 \textit{Build Robots and Make Them Move} needs dynamical models and feedback control, and both ROB 320 \textit{Robot Operating Systems} and ROB 330 \textit{Localization, Mapping, and Navigation} will benefit from the inclusion of inverse kinematics and constrained gradient descent, respectively. Furthermore, the foundational knowledge provided in this course will enhance a student's ability to excel in ROB 422 \textit{Introduction to Algorithmic Robotics} and ROB 489-002/3 \textit{Robot Control}.

\subsubsection{Integration with Michigan Robotics' Values}

% A core value of this course is its inclusivity and accessibility. Calculus often poses a significant barrier to entry for women and non-majority males, impacting their confidence and representation in engineering fields. By grounding calculus in computation and real-world applications, this course aims to demystify the subject and make it more approachable for all students.

This course, rooted deeply in the values of the Robotics Department, champions inclusivity and accessibility as its core principles. It is first of all important to recognize that Calculus often serves as a daunting barrier, particularly for women and underrepresented minorities, inadvertently hindering their progression within the engineering discipline. Furthermore, the disparity in high school mathematics preparation compounds the challenge of mastering calculus, affecting students' confidence and, ultimately, their election to pursue a STEM field. By integrating calculus with computational methods and real-world applications, this course is meticulously designed to dismantle these barriers. It sets reasonable expectations, elucidates concepts with clarity, and ensures that all students, regardless of their race, gender, economic, or educational background, find the subject equitable and approachable. Through this means, we are committed to making calculus not just a subject to be learned, but a gateway to empowerment and success in engineering for every student.

\subsubsection{Prerequisites and Future Pathways}

The only prerequisite for this course is ROB 101, ensuring that students are familiar with the Julia programming language and have experienced the practical applications of mathematics. To round out our Michigan students' mathematical education, they must earn 8 credits of mathematics in addition to \textit{Calculus for the Modern Engineer}. If they already have credit for Calc I, Calc II, or Calc IV before taking Calculus for the Modern Engineer, they can still count them. To take any of these courses after completing \textit{Calculus for the Modern Engineer} requires permission from an advisor. \hypertarget{CuratedListCourses}{Students at Michigan are urged to further their mathematical education with additional courses from a \textcolor{blue}{\bf carefully curated list}, including Math 215 {\it Multivariable and Vector Calculus}, Math 217 {\it Linear Algebra (with proofs)}, Math 312 {\it Applied Modern Algebra}, Math 351 {\it Principles of Analysis}, Math 371 (Engin 371) {\it Numerical Methods}, Math 412 {\it Abstract Algebra}, Math 416 {\it Theory of Algorithms}, and Math 451 {\it Advanced Calculus I}, a proof-based version of Calculus taught through the lens of Real Analysis in $\real^n$. In Robotics, mathematical knowledge is power, especially when you have the experience of rapidly turning it into code. 
 This approach complements traditional calculus courses, adding depth and breadth to our students' mathematical toolkits.}

\subsubsection{Why Julia?}

This textbook proudly uses Julia for its computational elements, embracing its open-source nature and cost-free availability as fundamental to our commitment to accessibility. Unlike the commonly used MATLAB, which comes with substantial licensing fees, Julia ensures that all students, regardless of their university's resources, can freely access state-of-the-art computational tools. This choice reflects our commitment to leveling the educational playing field, allowing every student to explore and excel in engineering without financial barriers.

\subsubsection{A New Paradigm in Calculus Education}

This textbook is not just a guide to calculus theory; it is a manual for applying calculus in the real world of engineering. It balances the need for theoretical understanding, computational skills, and practical application. While some manual calculations are included to bridge the gap with traditional calculus education, the emphasis is on programming-based solutions to complex, real-life engineering problems. This approach not only enhances understanding but also provides a clear, verifiable, and efficient method for problem-solving.

In summary, this textbook is more than just a collection of mathematical principles; it is a blueprint for a new way of teaching and learning calculus. It is designed not just to educate, but to inspire, opening doors to a world where calculus is not merely understood but applied with confidence and creativity.

\subsubsection{Dependency on ROB 101 \textit{Computational Linear Algebra}}

By building on the foundation laid in \href{https://robotics.umich.edu/academics/courses/course-offerings/rob101-fall-2020/}{ROB 101 \textit{Computational Linear Algebra}}, students entering this course have already experienced the power of mathematics in solving real-world engineering problems, such as analyzing extensive LiDAR datasets and creating a control algorithm to balance a planar model of a Segway. This background supports the innovative approach taken in this textbook, where programming and practical application are central to understanding and utilizing calculus in engineering contexts. If ROB 101 is not assumed, then the instructor should add a Julia programming lab to the course, perhaps modifying the (open-source) \href{https://github.com/michiganrobotics/ROB-101-Julia-Programming-Guide}{manual} created for ROB 101.

\subsubsection{Acknowledgement}
This textbook is dedicated to the group of students who were brave enough to take the pilot offering of ROB 201, \textit{Calculus for the Modern Engineer: Math at the Scale of Life}, in Fall 2024. It is also in honor of the undergraduate Instructional Assistants, Kaylee Johnson (lead),  Advaith (Adi) Balaji, Madeline (Maddy) Bezzina, Justin Boverhof, Elaina Mann, Reina Mezher, and Maxwell (Max) West, who collaborated with me in Winter and Summer 2024, diligently working to create homework sets, projects, and refine the initial draft of the textbook. Their invaluable insights and feedback have been instrumental in shaping this work. I am forever indebted to all of you.

ROB 201 was conceived by Prof. Chad Jenkins and the author as part of a complete undergraduate curriculum in robotics. Chad and I both thank past Associate Dean for Undergraduate Education, Prof. Joanna Mirecki Millunchick, for her encouragement in rethinking how mathematics is taught. 

\textbf{Jessy Grizzle}\\
Ann Arbor, Michigan USA, August 2024

